%%
%% This is file `sample-acmsmall.tex',
%% generated with the docstrip utility.
%%
%% The original source files were:
%%
%% samples.dtx  (with options: `acmsmall')
%%
%% IMPORTANT NOTICE:
%%
%% For the copyright see the source file.
%%
%% Any modified versions of this file must be renamed
%% with new filenames distinct from sample-acmsmall.tex.
%%
%% For distribution of the original source see the terms
%% for copying and modification in the file samples.dtx.
%%
%% This generated file may be distributed as long as the
%% original source files, as listed above, are part of the
%% same distribution. (The sources need not necessarily be
%% in the same archive or directory.)
%%
%%
%% Commands for TeXCount
%TC:macro \cite [option:text,text]
%TC:macro \citep [option:text,text]
%TC:macro \citet [option:text,text]
%TC:envir table 0 1
%TC:envir table* 0 1
%TC:envir tabular [ignore] word
%TC:envir displaymath 0 word
%TC:envir math 0 word
%TC:envir comment 0 0
%%
%%
%% The first command in your LaTeX source must be the \documentclass
%% command.
%%
%% For submission and review of your manuscript please change the
%% command to \documentclass[manuscript, screen, review]{acmart}.
%%
%% When submitting camera ready or to TAPS, please change the command
%% to \documentclass[sigconf]{acmart} or whichever template is required
%% for your publication.
%%
%%
\documentclass[acmsmall]{acmart}
\usepackage{ottalt}
\usepackage{xspace}
\usepackage[para]{footmisc}
\newcommand{\dotv}[2]{\href{#1}{\texttt{#1}}{\texttt{:#2}}}
\newcommand{\lang}{$\lambda^H$\xspace}
\inputott{rules}
%%
%% \BibTeX command to typeset BibTeX logo in the docs
\AtBeginDocument{%
  \providecommand\BibTeX{{%
    Bib\TeX}}}


%%
%% Submission ID.
%% Use this when submitting an article to a sponsored event. You'll
%% receive a unique submission ID from the organizers
%% of the event, and this ID should be used as the parameter to this command.
%%\acmSubmissionID{123-A56-BU3}

%%
%% For managing citations, it is recommended to use bibliography
%% files in BibTeX format.
%%
%% You can then either use BibTeX with the ACM-Reference-Format style,
%% or BibLaTeX with the acmnumeric or acmauthoryear sytles, that include
%% support for advanced citation of software artefact from the
%% biblatex-software package, also separately available on CTAN.
%%
%% Look at the sample-*-biblatex.tex files for templates showcasing
%% the biblatex styles.
%%

%%
%% The majority of ACM publications use numbered citations and
%% references.  The command \citestyle{authoryear} switches to the
%% "author year" style.
%%
%% If you are preparing content for an event
%% sponsored by ACM SIGGRAPH, you must use the "author year" style of
%% citations and references.
%% Uncommenting
%% the next command will enable that style.
\citestyle{acmauthoryear}


%%
%% end of the preamble, start of the body of the document source.
\begin{document}

%%
%% The "title" command has an optional parameter,
%% allowing the author to define a "short title" to be used in page headers.
\title{A Short and Mechanized Consistency Proof for Dependent Type Theory}

%%
%% The "author" command and its associated commands are used to define
%% the authors and their affiliations.
%% Of note is the shared affiliation of the first two authors, and the
%% "authornote" and "authornotemark" commands
%% used to denote shared contribution to the research.
\author{Yiyun Liu}
\orcid{0009-0006-8717-2498}
\affiliation{%
  \institution{University of Pennsylvania}
  \city{Philadelphia}
  \country{USA}
}
\email{liuyiyun@seas.upenn.edu}

%%
%% By default, the full list of authors will be used in the page
%% headers. Often, this list is too long, and will overlap
%% other information printed in the page headers. This command allows
%% the author to define a more concise list
%% of authors' names for this purpose.
% \renewcommand{\shortauthors}{Trovato et al.}

%%
%% The abstract is a short summary of the work to be presented in the
%% article.
\begin{abstract}
Proof by logical relation is a powerful technique that has been used
to derive metatheoretic properties of type systems, such as
consistency and parametricity. While there exists a
plethora of introductory materials about logical relation in the
context of simply typed lambda calculus and System F, a streamlined
presentation of proof by logical relation for a dependently language
is lacking in comparison. In this paper, I present a short
consistency proof for a dependently typed language that contains a
rich set of features, including an infinite cumulative universe
hierarchy, booleans, and an intensional identity type. We have
fully mechanized the consistency proof using the Coq proof assistant
in under 1000 lines of code.
\end{abstract}

%%
%% The code below is generated by the tool at http://dl.acm.org/ccs.cfm.
%% Please copy and paste the code instead of the example below.
%%
%%
%% Keywords. The author(s) should pick words that accurately describe
%% the work being presented. Separate the keywords with commas.
\keywords{Logical Relation, Dependent Types, Logical Consistency, Coq}

%%
%% This command processes the author and affiliation and title
%% information and builds the first part of the formatted document.
\maketitle

\section{Introduction}
% Depending on its application, we care about certain metatheoretic
% properties about a type system. As a programming language, we may care
% about type soundness, which states that a well-typed never gets stuck
% during evaluation.
When a dependently type system is used as a program logic where terms
encode proofs, we want our type system to be logically consistent,
meaning that the empty type is not inhabited.

The consistency proofs of various dependently typed systems, including Martin-Lof's
type theory and the Calculus of Constructions, have long been
available in the literature. In particular, recent works such as \citet{nbeincoq},
\citet{decagda}, and \citet{martin-lof-a-la-coq} mechanize stronger
results related to the decidability of type conversion or type
checking of dependently typed systems, from which consistency can be
derived as a corollary.

The underlying technique of the forementioned
works is proof by logical relation,
which involves interpreting types as reducibility predicates,
representing sets of terms satisfying certain properties with respect
to the reduction relation. While the proof technique and the
consistency result for dependent types are both well-established,
there is a severe lack of rigorous and accessible material that shows
how proof by logical relation can be applied to dependently typed
systems.

Introductory materials about logical relations such as
\citet{skorstengaard2019introduction}, \citet{harper2016practical}
start from logical relations for simply typed languages, and
eventually extend the technique to build a relational model for System
F in order to derive
parametricity. \citet{harperkripke} gives a gentle introduction on how
logical relations for closed terms can be extended to talk about open
terms so we can derive properties such as normalization for well-typed
open terms. In short, the introductory texts about logical relations
cover systems and properties with varying degrees of complexity, from
simply typed to polymorphicly typed, logical predicate to logical
equivalence, closed terms to open terms.

The glaring gap here is the
lack of fully dependently typed systems where computations may appear
at the type-level. In fact, it is far from obvious why proof by
logical relation is even applicable to dependent types, since the type
may very well-be a computation that is yet to be evaluated.
While it is assuring that proof by logical relations for dependent
types is available in mechanized forms, % the key to address the
% complexities of dependent types is obscured in
\citet{nbeincoq,decagda,martin-lof-a-la-coq} all involve relational,
Kripke-style models that obscure the technique for addressing
type-level computation. In fact, the added complexivity is evident
from the size of their code base. For example, \citet{nbeincoq}
involves 22 thousand lines of Coq code.

The goal of this paper is to give a tutorial on proof by logical relation for
dependent types in a simple and digestable format. % The language specification will
% be given in Section~\ref{sec:spec}.
The system we present is small but not minimal. We want to cover
features that are only available or require special treatments under
dependent types%  without introducing complexities that would have
% existed in its simply typed variant
. We choose boolean
types over natural numbers as our base type for simplicity, but include an
intensional identity type in our type system as the interpretation
of identity types requires an extra side condition that is needed for
indexed family of types. Unlike \citet{nbeincoq,decagda,martin-lof-a-la-coq}
but similar to \citet{anand2014towards}, we include an infinite
hierarchy of universes since we want to not only have at least one
predicative universe level to support type-level computations, but
also avoid the unnecessary code duplication pointed out by
\citet{nbeincoq}.

Our consistency proof is short and fully mechanized. The proof scripts
involve less than 1000 lines of Coq code. In fact, what is even more
encouraging is that among the 1000
lines of Coq code, 400 lines are related to the specification of the
type system, semantics, and properties related to untyped lambda
terms. The semantic type soundness proof through logical relation
takes almost the same amount of code as our syntactic type soundness
proof!
Thanks to the conciseness of
the proof, we are able to present most of its details in
Sections~\ref{sec:logreldep} and \ref{sec:logrelproof}. Moreover, the
structure of our mechanization closely corresponds to the proof we
present in the text, enabling us to label at the footnote each lemma
directly to their counterpart in the proof script for the readers to
reference and validate.

The technique we use is most similar to the one from \citet{nbeincoq},
which leverages impredicativity to define the logical
relation. % In our presented proof, we use the informal
% language of sets, where the use of impredicativity feels more natural
% and less noticeable.
Rather than framing impredicativity as a
mechanism for encoding induction
recursion~\citep{induction-recursion}, a scheme in which logical
relation for dependent types is typically defined~\citep{todo}, we opt
for a direct explanation in informal language of sets, where
impredicativity manifests in the form of second-order logical
formulas and thus more intuitive to understand for readers with some
general mathematics background.

In Section~\ref{sec:logrelmech}, we discuss the details about our
mechanization, including our use of existing libraries such as
Autosubst for handling bindings and CoqHammer for general-purpose
automation. In Section~\ref{sec:discuss}, we extend our logical
relation to prove the stronger property that every well-typed term
has a normal form. Most interestingly, we show that extensions such
as $\eta$ rules for functions and normalization for open terms are
mostly orthogonal to the fact that the language is dependently
typed. In our mechanized proof, the overall
structure of the logical relation remains unchanged, and the additional
proof obligations are mostly related to properties about the untyped
lambda calculus, all of which can be derived through syntactic
means.
In other words, once we have established the base technique for
proof by logical relation for dependent types, we can factor out the
complexities of an extension the are not specifically related to dependent
types; such an extension, if desired, may be studied in the context of
a simply typed language and later ported into a dependently typed setting.

We hope our success at creating a short and mechanized proof for a
relatively feature-complete language will encourage future researchers
to leverage the tool of logical relation more often in mechanized
proofs for dependent types.

% A naive attempt at proving consistency through induction over the
% typing derivation would fail since the inductive hypothesis is not
% strong enough to derive the consistency result. Instead, one typically
% relies on the technique referred to as proof by logical relation to
% interpret types as reducibility predicates to strengthen the inductive
% hypothesis.




% This paper is specifically
% about establishing logical consistency for a fully dependently typed
% system with an infinite universe hierarchy and support for large
% elimination .
% The type system, presented in Section~\ref{sec:spec}, is
% most similar to Martin-Lof's predicate type theory with the minor
% difference that type conversion is based on untyped equality.


% Rather,
% our goal is to present the proof in a form that is digestable by a
% working type theorist and can be more readily mechanized in a proof
% assistant. Compared to existing efforts at mechanizing logical
% consistency or stronger properties such as existence of normal
% form~\citep{nbeincoq},
% decidable type checking~\citep{decagda}, our work is minimal since it requires very
% little scaffolding and therefore results in an extremely succinct
% proof of under 1000 lines of manually written Coq code for a dependent
% type theory that is reasonably complete in its features.

% The key technique that underlies our consistency proof is proof by
% logical relation. In
% Section~\ref{sec:spec}, we present the dependent type theory of
% interest. In Section~\ref{sec:logreldep}, we give the definition
% of the logical relation for the dependent type theory. Rather than
% presenting the logical relation as an inductive-recursive definition,
% we use the more elementary concept of a partial function to capture
% the interpretation of types. The alternative representation requires us
% to show that the set of equations indeed defines a partial function;
% that is, for each input, there should always be a unique
% output.
% From the interpretation function, we can define the semantic
% typing judgment for the set of lambda terms.
% In Section~\ref{sec:logrelproof}, we prove the fundamental theorem,
% which states that syntactic typing implies semantic typing. Once the
% fundamental theorem is established, logical consistency follows as a
% trivial corollary. In Section~\ref{sec:logrelmech}, we point out the
% specifics related to the Coq mechanization of the proof described in earlier
% section.
% Finally, in Section~\ref{sec:relatedwork}, we give a short survey of
% existing literature related to logical consistency about dependent
% type theory.

\section{Specification of a Dependent Type Theory}
\label{sec:spec}

\begin{figure}[h]
\[
\begin{array}{lcll}
\mathit{Natural\ numbers}\\
[[i]],[[j]],[[n]] & \in &  [[SNat]] &  \\ \\

\mathit{Contexts}\\
[[G]]       & ::= & [[empty]]\ |\ [[G ++ A]] &  \\ \\
\mathit{Terms}\\
[[a]],[[b]],[[c]],[[t]],[[p]],[[A]],[[B]] & ::= & [[Set i]]\ |\ [[i]]\  |\ [[Void]]
                  & \mbox{universes, variables, empty type} \\
            & |   & [[Pi A B]]\ |\ [[\ A a]]\ |\ [[a b]]
                  & \mbox{function types, abstractions, applications} \\
            & |   & [[a ~ b : A ]]\ |\  [[refl]]\ |\ [[J t a b p]]
                  & \mbox{equality types, reflexivity proof, J eliminator} \\
            & |   & [[Bool]]\ |\  [[true]]\ |\  [[false]]\ |\  [[if a b0 b1]]
                  & \mbox{boolean type, true, false, if} \\ \\
\mathit{Renaming}\\
[[xi]] & \in & [[SNat -> SNat]] & \\ \\
\mathit{Substitution}\\
[[rho]] & \in & [[SNat -> STm]] &
\end{array}
\]
  \caption{Syntax of \lang}
  \label{fig:syntax}
\end{figure}

\begin{figure}[h]
    \[
      \begin{array}{lll}
        \mathit{Curried\ Addition} \\
        add(n) & := & m \mapsto n + m \\ \\
        \mathit{Extension ([[xi]])} \\
        [[(xi .: j) 0]]  & := & 0 \\
        [[(xi .: j) Suc i]]  & := & [[xi j]] \\ \\

        \mathit{Extension ([[rho]])} \\
        [[(rho .: a) 0]]  & := & [[a]] \\
        [[(rho .: a) Suc i]]  & := & [[rho i]] \\ \\

        \mathit{Up ([[xi]])} \\
        [[up xi]] & := & (add(1) \circ [[xi]]) , 0 \\ \\

        \mathit{Renaming} \\
        [[i {xi}]] & := & [[xi i]] \\
        [[(Set i) {xi}]] & := & [[Set i]] \\
        [[Void {xi}]] & := & [[Void]]\\
        [[(Pi A B) {xi}]] & := & [[Pi A{xi} B{up xi}]] \\
        [[(\ A a) {xi}]] & := & [[\ A {xi} (#a {up xi}#)]] \\
        [[(a b) {xi}]] & := & [[a {xi} (# b {xi} #)]] \\
        [[Bool {xi}]] & := & [[Bool]] \\
        [[true {xi}]] & := & [[true]] \\
        [[false {xi}]] & := & [[false]] \\
        [[(if a b0 b1) {xi}]] & := & [[if a{xi} b0{xi} b1{xi}]] \\
        [[(a ~ b : A) {xi}]] & := & [[ a{xi} ~ b{xi} : A{xi}]] \\
        [[refl {xi}]] & := & [[refl]] \\
        [[(J t a b p ) {xi}]] & := & [[J t {xi} a {xi} b {xi} p{xi}]] \\ \\

        \mathit{Shift} \\
        [[up n A]] & := & A \langle add(n) \rangle \\ \\

        \mathit{Shift\ one} \\
        [[up1 A]] & := & \uparrow^1 [[A]] \\ \\

        \mathit{Lookup} \\
        [[(G ++ A) 0]] & := &  [[A]] \\
        [[(G ++ A) Suc i]] & := & [[G i]] \\ \\

        \mathit{Drop} \\
        [[drop 0 G]] & := & [[G]] \\
        [[drop Suc i (#G ++ A#)]] & := & [[drop i G]] \\ \\
      \end{array}
    \]
  \caption{Auxiliary Functions over Syntax}
  \label{fig:auxdef}
\end{figure}


\begin{figure}[h]
\drules[P]{$[[a => b]]$}{Parallel Reduction}{Var, Set, Void, Pi, Abs, App, AppAbs, True, False, If, IfTrue, IfFalse, Bool, Eq, Refl, J, JRefl}
\drules[PS]{$[[a =>+ b]]$}{Transitive Closure of Parallel Reduction}{One, Step}
\drules[C]{$[[a <=> b]]$}{Coherence}{Intro}
\caption{Parallel reduction and coherence}
\label{fig:par}
\end{figure}

\begin{figure}[h]
\drules[Ctx]{$[[ |- G]]$}{Context Well-Formedness}{Empty, Cons}
\drules[T]{$[[G |-  a : A]]$}{Typing}{Var, Set, Pi, Abs, App, Conv, Refl, J, Bool, True, False, If, Void}
\caption{Syntactic typing for \lang}
\label{fig:typing}
\end{figure}


In this section, we present the dynamics and statics of the
dependent type theory whose logical consistency will be proven in
Section~\ref{sec:logrelproof}. For concision, we refer to this system
as \lang.

The syntax of \lang can be found in Figure~\ref{fig:syntax}. We will use
the unscoped de Bruijn representation for the rest of the paper,
adopting notations from \citet{autosubst2}, summarized in Figure~\ref{fig:auxdef}.
Without providing
the full definition of the renaming and substitution functions, it is
impossible to tell the binding structure from the syntax
alone. Therefore, we annotate the syntax in Figure~\ref{fig:syntax}
with the de Bruijn depth of each term, though we note that the
syntax we work with is unscoped and the choice does matter when
we extend our logical relation to open terms in Section~\ref{sec:discuss}.

% Figure~\ref{fig:auxdef} shows the auxiliary definitions over the term
% syntax, including renaming, substitution, and operations over
% the typing context or substitution.
As a
dependent type theory, terms and types are collapsed into the same
syntactic category. Dependent functions take the form $[[Pi A B]]$ and
we use the notation $[[A -> B]]$ when the output type $[[B]]$ is not
dependent on the input variable. $[[Set i]]$ represents the universe
type where $[[i]]$ ranges over the set of natural numbers.
 Finally,
\lang also includes an intensional identity type $[[a ~ b : A]]$ whose
proofs can be eliminated by the J-eliminator $[[J t a b p]]$, where
$[[p]]$ is an equality proof between $[[a]]$ and $[[b]]$, and $[[t]]$
is the term whose type is to be casted.

\lang is expressive enough to support large
elimination, the ability to compute a type using a term as input. For
example, the function $[[\ Bool if 0 Bool Bool -> Bool]]$ returns
either $[[Bool]]$ or $[[Bool -> Bool]]$ depending on whether the input
is $[[true]]$ or $[[false]]$.

Figure~\ref{fig:par} shows the definition of the parallel reduction
relation, which takes the form $[[a => b]]$. We use $[[a =>+ b]]$ to
represent its transitive and reflexive closure, which in turn allows us to define
the coherence relation $[[a <=> b]]$.
We say that two terms $[[a]]$
and $[[b]]$ are coherent if they can eventually reduce to some common
term $[[c]]$ through parallel reduction. The symmetric notation of
coherence suggests that it is an equivalence relation, and in fact, we
will use coherence as the equational theory for our type conversion
rule.

We sketch out some key properties about coherence without giving their
proofs. Our technique for establishing those results is based on
\citet{takahashi-parallel-reduction} and a modern exposition of the
same technique can be found in \citet{plfa22.08}.

TODO: remove the lengthy discussion below

Now, we prove that coherence is indeed an equivalence relation.

First, we show that coherence is reflexive through the following
sequence of lemmas.
\begin{lemma}[Par Refl\footnote{\dotv{join.v}{Par\_refl}}]
  \label{lemma:parrefl}
  For all terms $[[a]]$, $[[a => a]]$.
\end{lemma}

\begin{lemma}[Pars refl]
  \label{lemma:parsrefl}
  For all terms $[[a]]$, $[[a =>+ a]]$.
\end{lemma}

\begin{lemma}[Coherence refl\footnote{\dotv{join.v}{Coherent\_reflexive}}]
  \label{lemma:coherencerefl}
  For all terms $[[a]]$, $[[a <=> a]]$.
\end{lemma}

Lemma~\ref{lemma:parrefl} can be proven by structural induction over
the term $[[a]]$. Lemmas~\ref{lemma:parsrefl} and \ref{lemma:coherencerefl}
immediately follow as corollaries of Lemma~\ref{lemma:parrefl}.

The reflexivity of parallel reduction enables us to embed rules from
call-by-name semantics into parallel reduction, as the following lemma
shows.
\begin{lemma}[Par AbsCbn\footnote{\dotv{join.v}{P\_AppAbs\_cbn}}] For all $[[A]], [[a]],$ and $[[b]]$,
  \label{lemma:parabscbn}
  $[[(\ A a) b => b {a}]]$
\end{lemma}
\begin{proof}
  Immediate from Lemma~\ref{lemma:parrefl} and \rref{P-AppAbs}.
\end{proof}

Symmetry of coherence immediately falls from its definition.
\begin{lemma}[Coherence sym\footnote{\dotv{join.v}{Coherent\_symmetric}}]
  \label{lemma:coherencesym}
  If $[[a <=> b]]$, then $[[b <=> a]]$.
\end{lemma}

Before we can prove transitivity, we need to show that parallel
reduction satisfies the diamond property.
\begin{lemma}[Par cong\footnote{\dotv{join.v}{par\_cong}}]
  \label{lemma:parcong}
  If $[[a0 => a1]]$ and $[[b0 => b1]]$, then $[[a0 { b0 } => a1 { b1 }]]$.
\end{lemma}
\begin{lemma}[Par diamond\footnote{\dotv{join.v}{par\_confluent}}]
  \label{lemma:pardiamond}
  If $[[a => b0]]$ and $[[a => b1]]$, then there exists some term
  $[[c]]$ such that $[[b0 => c]]$ and $[[b1 => c]]$.
\end{lemma}
The congruence property (Lemma~\ref{lemma:parcong}) can be proven by
structural induction over the derivation of $[[a0 => a1]]$.
Likewise, Lemma~\ref{lemma:pardiamond} can be proven by structural induction
over the derivation of $[[a => b0]]$. The \rref{P-AppAbs} case requires
the use of Lemma~\ref{lemma:parcong}.

From Lemma~\ref{lemma:parcong} and \ref{lemma:parrefl}, we recover the
single substitution property as a simple corollary.
\begin{corollary}[Par subst\footnote{\dotv{join.v}{par\_subst}}]
  \label{lemma:parsubst}
  If $[[a0 => a1]]$, then $[[a0 {b} => a1 {b}]]$ for arbitrary $[[b]]$.
\end{corollary}


A relation that satisfies the
diamond property must also be confluent, meaning that its transitive
and reflexive closure is confluent.
\begin{lemma}[Par confluent\footnote{\dotv{join.v}{pars\_confluent}}]
  \label{lemma:parconfluent}
  If $[[a =>+ b0]]$ and $[[a =>+ b1]]$, then there exists some term
  $[[c]]$ such that $[[b0 =>+ c]]$ and $[[b1 =>+ c]]$.
\end{lemma}
While $[[a =>+ b]]$ is defined as the transitive closure of $[[a => b]]$,
it coincides with the transitive and reflexive closure of $[[a => b]]$ since $[[a
=> b]]$ is reflexive (Lemma~\ref{lemma:parrefl}).

The transitivity of the coherence relation follows as a corollary of
Lemma~\ref{lemma:parconfluent}.
\begin{lemma}[Coherence trans\footnote{\dotv{join.v}{Coherent\_transitive}}]
  \label{lemma:coherencetrans}
  If $[[a0 <=> a1]]$ and $[[a1 <=> a2]]$, then $[[a0 <=> a2]]$.
\end{lemma}
By the definition of coherence, there exists some term $[[b]]$ such that $[[a0 =>+ b0]]$,
$[[a1 =>+ b0]]$ and some term $[[b1]]$ such that $[[a1 =>+ b1]]$ and
$[[a2 =>+ b1]]$. By Lemma~\ref{lemma:parconfluent}, there exists some
term $[[c]]$ such that $[[b0 =>+ c]]$ and $[[b1 =>+ c]]$. It sufficies
to show that $[[a0 =>+ c]]$ and $[[a2 =>+ c]]$, both of which
trivially hold since the transitive closure $[[a =>+ b]]$ is transitive.
This concludes the proof that coherence is an equivalence relation.
\begin{lemma}[Coherence Equivalence]
  \label{lemma:coherenceequiv}
  The relation $[[a <=> b]]$ satisfies reflexivity, symmetry, and
  transitivity and therefore is an equivalence relation.
\end{lemma}

Figure~\ref{fig:typing} gives the full typing rules for
\lang{}. \Rref{T-Conv} uses the definition of coherence from earlier
for type conversion. The use of an untyped relation for type
conversion makes our formulation slightly different from languages
such as MLTT, where the judgmental equality typically takes the form
TODO, where terms ... and ... are known to be well-typed. The
equivalence of such systems and a system that uses untyped equality
are explored in detail in ...

Without fancy eta laws, it is easy to embed a typed language into an
untyped language.


% We note that a more conventional presentation of
% \rref{T-Conv} would instead use full beta reduction as the base for
% the definition of coherence. However, since full beta reduction
% doesn't satisfy the diamond property, one typically needs parallel
% reduction as an auxilliary definition to derive the confluence of full
% beta reduction. Our formulation of \lang through parallel reduction
% is slightly more economical.

\section{Logical Relation}
\label{sec:logreldep}
\begin{figure}[h]
\drules[I]{$[[Interp I i A S]]$}{Logical Relation}{Void, Bool, Eq, Pi, Set, Red}
\caption{Logical relation for \lang}
\label{fig:logrel}
\end{figure}
In this section, we define the logical relation for \lang{} in the
form of an inductively defined relation (Figure~\ref{fig:logrel}). The
logical relation takes the form $[[Interp I i A S]]$. The
metavariables $[[A]]$ and $[[i]]$ stand for terms and natural
numbers respectively, as introduced earlier in
Figure~\ref{fig:syntax}.
The metavariables $[[I]]$ and $[[S]]$ are
sets with the following signatures:
\begin{equation*}
  \begin{split}
    [[I]] &\in [[ { j | j < i  } ->  PowerSet STm ]] \\
    [[S]] &\in [[PowerSet STm]]
  \end{split}
\end{equation*}
The function $[[I]]$ is a family of subsets of terms indexed by
natural numbers strictly less than the parameter $[[i]]$, which
represents the current universe level.  In \rref{I-Set}, function
$[[I]]$ is used to define the meaning of
universes that are strictly smaller than the current level $[[i]]$. The
restriction $[[j < i]]$ in \rref{I-Set} is crucial for our system to
be predicative. Removing the ordering constraint would result in a
system where one can encode Girard's paradox~\citep{girard-thesis}.

\begin{figure}[h]
\begin{equation*}
    [[InterpR i A S]] := [[ Interp I i A S  ]], \text{where } [[I i]] := [[{A | exists S , InterpR i A S}]]
\end{equation*}
\caption{Logical relation for all universe levels}
\label{fig:logrelrec}
\end{figure}
To tie the knot and obtain an interpretation of all universe levels,
we define in Figure~\ref{fig:logrelrec} the final version of our interpretation judgment recursively
using the well-foundedness of the strict less than relation on natural
numbers (recall that
the parameter $[[I]]$ of $[[Interp I i A S]]$ takes only natural
numbers strictly less than $[[i]]$ as its input).
The judgment $[[InterpR i A S]]$ now reads that the type $[[A]]$ is a
level-$[[i]]$ type \emph{semantically} inhabited by terms from the set
$[[S]]$. For the
majority of the properties we are about to prove in this section, we
do not need any information about the parameterized function $[[I]]$.
Each property about $[[InterpR i A S]]$ follows as a corollary of
a property about $[[Interp I i A S]]$ with no or few assumptions imposed on
$[[I]]$. As a result, we usually state our lemmas in terms of
$[[Interp I i A S]]$ without duplicating them in terms of $[[InterpR i
A S]]$.

For the rest of the section, we establish some important facts about
the logical relation that will be useful for proving fundamental theorem
in Section~\ref{sec:logrelproof}.

The relation $[[Interp I i A S]]$ satisfies the following inversion
principles.
\begin{lemma}[Inversion of the logical relation]
  \label{lemma:interpinv}\leavevmode
  \begin{enumerate}
  \item If $[[Interp I i Void S]]$, then $[[S = emptyset]]$.\footnote{\dotv{semtyping.v}{InterpExt\_Void\_inv}}
  \item If $[[Interp I i Bool S]]$, then $[[S = { a | a =>+ true \/ a =>+ false   }]]$.\footnote{\dotv{semtyping.v}{InterpExt\_Bool\_inv}}
  \item If $[[Interp I i a ~ b : A S]]$, then $[[S = { p | p =>+ refl , a <=> b  }]]$.\footnote{\dotv{semtyping.v}{InterpExt\_Eq\_inv}}
  \item If $[[Interp I i Pi A B S0]]$, then there exists $[[S]],[[F]]$ such that:\footnote{\dotv{semtyping.v}{InterpExt\_Fun\_inv}}
    \begin{itemize}
    \item $[[Interp I i A S ]]$
    \item $[[F in S -> PowerSet STm]]$
    \item $[[forall a, (# a in S implies Interp I i B { a } F a #)]]$
    \item $[[S0 = { b | forall a, (# a in S implies b a in F a #) }]]$
    \end{itemize}
  \item If $[[Interp I i Set j S]]$, then $[[j < i]]$ and $[[S = I j]]$.\footnote{\dotv{semtyping.v}{InterpExt\_Univ\_inv}}
  \end{enumerate}
\end{lemma}
\begin{proof}
  We only show the most involved function case; the rest follows a
  similar but simpler pattern. We start by
  inducting over the derivation of $[[Interp I i Pi A B S]]$. There
  are only two possible cases we need to consider.
  \begin{description}
  \item[\Rref{I-Pi}:] Immediate.
  \item[\Rref{I-Red}:] We are given that $[[Interp I i Pi A B S0]]$.
    We know that there exists some $[[A0]]$ and
    $[[B0]]$ such that $[[Pi A B => Pi A0 B0]]$ and $[[Interp I i Pi
    A0 B0 S0]]$. From the
    induction hypothesis, there exists $[[S]]$ and $[[F]]$ such that :
    \begin{itemize}
    \item $[[Interp I i A0 S ]]$
    \item $[[F in S -> PowerSet STm]]$
    \item $[[forall a, (# a in S implies Interp I i B0 { a } F a #)]]$
    \item $[[S0 = { b | forall a, (# a in S implies b a in F a #) }]]$
    \end{itemize}
    By inverting $[[Pi A B => Pi A0 B0]]$, we derive $[[A => A0]]$ and
    $[[B => B0]]$. By Lemma~\ref{lemma:parsubst}, we have $[[B {a} => B0 {a} ]]$ for all
    $[[a]]$. As a result, by \rref{I-Red}, the same $[[S]]$ and
    $[[F]]$ satisfies the following properties:
    \begin{itemize}
    \item $[[Interp I i A S ]]$
    \item $[[F in S -> PowerSet STm]]$
    \item $[[forall a, (# a in S implies Interp I i B { a } F a #)]]$
    \end{itemize}
    The properties above are exactly the preconditions needed to apply
    \rref{I-Pi} for $[[Pi A B]]$ to finish off the proof.
  \end{description}
\end{proof}

\Rref{I-Red} bakes into the logical relation the backward preservation
property. The following property shows that preservation holds in the
usual forward direction too.
\begin{lemma}[Preservation of the logical relation\footnote{\dotv{semtyping.v}{InterpExt\_preservation}}]
  \label{lemma:interppreservation}
  If $[[Interp I i A S]]$ and $[[A => B]]$, then $[[Interp I i B S]]$.
\end{lemma}
\begin{proof}
  We carry out the proof by induction over the derivation of $[[Interp
  I i A S]]$.
  \begin{description}
  \item[\Rref{I-Void}:] There exists some $[[B]]$ such that $[[Void =>
    B]]$. By inverting the derivation of $[[Void => B]]$, $[[B]]$ must
    be $[[Void]]$ and the result trivially follows.
  \item[\Rref{I-Bool, I-Set}:] Similar to the case for \rref{I-Void}.
  \item[\Rref{I-Eq}:] We know that $[[Interp I i a ~ b : A { p | p =>+
      refl , a <=> b  }]]$ and, by inverting the derivation of
    parallel reduction, $[[a => a0]]$, $[[b => b0]]$, $[[A => A0]]$
    for some $[[a0]]$, $[[b0]]$, and $[[A0]]$. Our goal is to show
    that $[[Interp I i a0 ~ b0 : A {p | p =>+ refl, a <=> b}]]$. By
    \rref{I-Eq}, we already know that $[[Interp I i a0 ~ b0 : A {p | p
      =>+ refl, a0 <=> b0}]]$ and therefore it suffices to show that
    the sets $[[{p | p =>+ refl, a <=> b}]]$ and $[[{p | p =>+ refl,
      a0 <=> b0}]]$ are equal. Equivalently, it suffices to show that
    $[[a <=> b]]$ if and only if $[[a0 <=> b0]]$. By definition, from
    $[[a => a0]]$ and $[[b => b0]]$, we derive $[[a <=> a0]]$ and $[[b
    <=> b0]]$. The result then immediately follows from the fact that
    coherence is an equivalence relation
    (Lemma~\ref{lemma:coherenceequiv}).
  \item[\Rref{I-Pi}:] There exists $[[S]]$ and $[[F]]$
    such that:
    \begin{itemize}
    \item $[[Interp I i Pi A B { b | forall a, (# a in S implies b a in F
        a #) }]]$
    \item $[[Interp I i A S ]]$
    \item $[[F in S -> PowerSet STm]]$
    \item $[[forall a, (# a in S implies Interp I i B { a } F a #)]]$
    \end{itemize}
    There exists some $[[A0]]$ and $[[B0]]$ such that $[[A => A0]]$ and
    $[[B => B0]]$. Our goal is to show that $[[Interp I i Pi A0 B0 { b | forall a, (# a in S implies b a in F
      a #) }]]$. By \rref{I-Pi}, it suffices to show:
    \begin{itemize}
    \item $[[Interp I i A0 S ]]$
    \item $[[F in S -> PowerSet STm]]$
    \item $[[forall a, (# a in S implies Interp I i B0 { a } F a #)]]$
    \end{itemize}
    Since $[[A => A0]]$ and $[[B {a} => B0 {a}]]$ forall $[[a]]$, the
    latter of which follows from Lemma~\ref{lemma:parsubst}, the above
    conditions follow immediately from the induction hypothesis.
  \item[\Rref{I-Red}:] There exists $[[B0]]$ and $[[S]]$ such that $[[A => B0]]$ and
    $[[Interp I i B0 S]]$. Given an arbitrary $[[B]]$ such that $[[A
    => B]]$, our goal is to show that $[[Interp I i B S]]$. By the diamond
    property of parallel
    reduction (Lemma~\ref{lemma:pardiamond}), there exists some term
    $[[C]]$ such that $[[B0 => C]]$ and $[[B => C]]$. By the induction
    hypothesis, we derive $[[Interp I i C S]]$ from $[[Interp I i
    B0 S]]$. By \rref{I-Red} and
    $[[B => C]]$, we conclude that $[[Interp I i B S]]$.
  \end{description}
\end{proof}
From Lemma~\ref{lemma:interppreservation} and \rref{I-Red}, we can easily
derive the following corollary that two coherent types have the same
interpretation.
\begin{corollary}[Logical Relation for Coherent Types\footnote{\dotv{semtyping.v}{InterpUnivN\_Coherent}}]
  \label{lemma:logrelcoherence}
  If $[[Interp I i A S]]$ and $[[A <=> B]]$, then $[[Interp I i B S]]$.
\end{corollary}
Next, we show that $[[Interp I i A S]]$ is in fact a partial function.
\begin{lemma}[Logical relation is functional\footnote{\dotv{semtyping.v}{InterpExt\_deterministic}}]
  \label{lemma:logreldeter}
  If $[[Interp I i A S0]]$ and $[[Interp I i A S1]]$, then $[[S0 = S1]]$.
\end{lemma}
\begin{proof}
  We start by inducting over the derivation of the first premise $[[Interp I i A
  S0]]$.
  \begin{description}
  \item[\Rref{I-Void}:] We know that $[[Interp I i Void
    emptyset]]$. Given $[[Interp I i Void S1]]$, our goal is to show that
    $[[emptyset = S1]]$. This is immediate by applying the $[[Void]]$ case of
    Lemma~\ref{lemma:interpinv} to $[[Interp I i Void S1]]$.
  \item[\Rref{I-Bool, I-Eq, I-Set}:] Similar to the \rref{I-Void} case
    by applying the matching case of Lemma~\ref{lemma:interpinv}
    to $[[Interp I i A S1]]$.
  \item[\Rref{I-Pi}:] There exists $[[S]]$ and $[[F]]$
    such that:
    \begin{itemize}
    \item $[[Interp I i A S ]]$
    \item $[[F in S -> PowerSet STm]]$
    \item $[[forall a, (# a in S implies Interp I i B { a } F a #)]]$
    \end{itemize}
    Our goal is to show that given $[[Interp I i Pi A B S1]]$, we have
    $[[S1 = { b | forall a, (# a in S implies b a in F
      a #) }]]$.
    By the function case of Lemma~\ref{lemma:interpinv}, there exists
    some $[[S0]]$ and $[[F0]]$ such that:
    \begin{itemize}
    \item $[[Interp I i A S0 ]]$
    \item $[[F0 in S0 -> PowerSet STm]]$
    \item
      $[[forall a, (# a in S0 implies Interp I i B { a } F0 a #)]]$
    \item
      $[[S1 = { b | forall a, (# a in S0 implies b a in F0 a #) }]]$
    \end{itemize}
    It suffices to show that $[[S = S0]]$ and $[[F = F0]]$. The
    equality $[[S = S0]]$ immediately follows from the induction
    hypothesis since $[[Interp I i A S]]$ and $[[Interp I i A
    S0]]$. Therefore, functions $[[F]]$ and $[[F0]]$ have the same
    domain and codomain and thus it suffices to show that forall $[[a in S]]$,
    $[[F a = F0 a]]$. Suppose $[[a in S]]$, we must have $[[Interp I
    i B {a} F a]]$ and $[[Interp I i B {a} F0 a]]$ from the two
    $\forall$-quantified statements above. The equality $[[F a = F0 a]]$ then
    immediately follows from the induction hypothesis.
  \item[\Rref{I-Red}:] There exists some $[[B]]$ such that $[[A =>
    B]]$ and $[[Interp I i B S0]]$. Our goal is to show that given
    $[[Interp I i A S1]]$, we have $[[S0 = S1]]$. By
    Lemma~\ref{lemma:interppreservation}, from $[[Interp I i A S1]]$
    and $[[A => B]]$, we have $[[Interp I i B S1]]$. From the
    induction hypothesis, we can conclude that $[[S0 = S1]]$ since
    $[[Interp I i B S0]]$ and $[[Interp I i B S1]]$.
  \end{description}
\end{proof}

Lemma~\ref{lemma:logreldeter} enables us to derive the following
alternative introduction and inversion principles for \rref{I-Pi}.
\begin{lemma}[Pi Intro Alt\footnote{\dotv{semtyping.v}{InterpExt\_Fun\_nopf}}]
  \label{lemma:piintroalt}
  Suppose the following two statements hold:
  \begin{itemize}
  \item $[[Interp I i A S]]$
  \item $[[forall a, a in S implies (# exists S0 , Interp I i B {a} S0 #)]]$
  \end{itemize}
  Then we have $[[Interp I i Pi A B { b | forall a, (# a in S , forall
    S0, (# Interp I i B {a} S0,  b a in F a #) #) }]]$
\end{lemma}
\begin{proof}
  Let $[[F]]$ be the relation defined as follows:
  \[ (a,S) \in [[F]] \iff [[Interp I i B {a} S]] \]
  By the second bullet from the premise and
  Lemma~\ref{lemma:logreldeter}, $[[F]]$ is a function that is total
  on the set $[[S]]$. The conclusion then trivially follows from
  \rref{I-Pi}.
\end{proof}

\begin{lemma}[Pi Inv Alt\footnote{\dotv{semtyping.v}{InterpExt\_Fun\_inv\_nopf}}]
  \label{lemma:piinvalt}
  Suppose $[[Interp I i Pi A B S]]$, then there exists some $[[S0]]$
  such that the following constraints hold:
  \begin{itemize}
  \item $[[Interp I i A S0]]$
  \item $[[forall a, (# a in S0 implies (# exists S1 , Interp I i B {a}
    S1 #) #)]]$
  \item $[[S = { b | forall a, (# a in S0 , forall
    S1, (# Interp I i B {a} S1,  b a in S1 #) #) }]]$
  \end{itemize}
\end{lemma}
\begin{proof}
  Immedaite from Lemmas~\ref{lemma:interpinv} and \ref{lemma:logreldeter}.
\end{proof}

% The combination of Lemmas~\ref{lemma:piintroalt} and
% \ref{lemma:piinvalt} reveals what we truly want to capture with
% \rref{I-Pi}. A function type is semantically well-defined if its input
% type is well-defined, and its output type is defined for each valid
% inahbitant of its input type.

The next lemma shows that our logical relation satisfies
cumulativity. That is, if a type has an interpretation at a lower
universe level, then we can obtain the same interpretation at a higher
universe level.
\begin{lemma}[Logical relation cumulativity\footnote{\dotv{semtyping.v}{InterpExt\_cumulative}}]
  \label{lemma:logrelcumulativity}
  If $[[Interp I i0 A S]]$ and $[[i0 < i1]]$, then $[[Interp I i1 A S]]$.
\end{lemma}
\begin{proof}
  Trivial by structural induction over the derivation of $[[Interp I
  i0 A S]]$.
\end{proof}
Note that in the statement of Lemma~\ref{lemma:logrelcumulativity}, we
implicitly assume that $[[I]]$ is defined on the set of natural
numbers less than $[[i1]]$.

\begin{corollary}[Logical relation is functional with different levels\footnote{\dotv{semtyping.v}{InterpExt\_deterministic'}}]
  \label{lemma:logreldeterhet}
  If $[[Interp I i0 A S0]]$ and $[[Interp I i1 A S1]]$, then $[[S0 = S1]]$.
\end{corollary}
\begin{proof}
  Immediate from Lemma~\ref{lemma:logreldeter} and
  \ref{lemma:logrelcumulativity}.
\end{proof}

We say that a set of terms $[[S]]$ is closed under expansion if given
$[[a in S]]$, then $[[b in S]]$ for all $[[b => a]]$.
The final property we want to show is that the output set $[[S]]$ from
the logical relation is closed under expansion. Unlike the previous
lemmas, we need to constrain the function $[[I]]$ so its outputs are
all closed under expansion.
\begin{lemma}[Logical Relation Elements back preservation\footnote{\dotv{semtyping.v}{InterpExt\_back\_clos}}]
  \label{lemma:logrelbackclos}
  If $[[Interp I i A S]]$ and $[[I]]$ satisfies the property that
  for all $[[i]]$, $[[I i]]$ is closed under expansion, then the set
  $[[S]]$ is closed under expansion.
\end{lemma}
\begin{proof}
  Trivial by structural induction over the derivation of $[[Interp I i
  A S]]$. The function case requires the following simple fact about
  parallel reduction:
  If $[[b0 => b1]]$ then $[[b0 a => b1 a]]$ for all $[[a]]$. This fact
  is not immediate from the definition of parallel reduction but
  follows from Lemma~\ref{lemma:parrefl} and \rref{P-App}.
\end{proof}

\begin{corollary}[Logical Relation Elements back preservation (rec)\footnote{\dotv{semtyping.v}{InterpUnivN\_back\_clos}}]
  \label{lemma:logrelNbackclos}
  If $[[InterpR i A S]]$, then $[[S]]$ is closed under expansion.
\end{corollary}
\begin{proof}
  Immediate from Lemma~\ref{lemma:logrelbackclos}, the definition of
  $[[InterpR i A S]]$, and \rref{I-Red}.
\end{proof}

\section{Semantic Typing and Consistency}
\label{sec:logrelproof}
\begin{figure}[h]
  \begin{equation*}
    \begin{split}
      [[rho |= G]] &:= \forall i\ j\ S, \text{ if }[[i < |G|]]\text{ and
                     } [[InterpR j (up Suc i G i) { rho } S ]] \text{, then } [[rho i in S]] \\
      [[G |= a : A]] &:= \forall [[rho]], \text{ if }[[rho |=
                       G]]\text{ then there exists some } [[j]] \text{
                       and } [[S]] \text{ such that } [[InterpR j A
                       {rho} S]] \text{ and } [[a {rho} in S]] \\
      [[|= G]] &:= \forall [[i < |G|]], \exists [[j]], [[drop Suc i G |= G i : Set j]]
    \end{split}
  \end{equation*}
  \caption{Semantic Typing for \lang}
  \label{fig:semtyping}
\end{figure}


The logical relation we define in Figure~\ref{fig:logrel} does not
include cases for variables. Likewise, for the base types such as
boolean and equality, the output set $[[S]]$ contains only terms that
evaluate to closed terms. To generalize our logical relation to open
terms, we define the semantic typing judgment by closing the open
terms with a substitution whose codomain consists of terms that
respect the interpretation of the types from the context. The full
definitions of well-formed substitution ($[[rho |= G]]$), semantic
typing ($[[ G |= a : A]]$), and semantic context well-formedness
($[[|= G]]$) are presented in Figure~\ref{fig:semtyping}.

The following lemma makes the statement $[[G |= A : Set i]]$ easier to
work with.
\begin{lemma}[Set Inv\footnote{\dotv{soundness.v}{SemWt\_Univ}}]
  \label{lemma:setinv}
  The following two statements are equivalent:
  \begin{itemize}
  \item $[[G |= A : Set i]]$
  \item For all $[[rho]]$, if $[[rho |= G]]$, then there exists
    $[[S]]$ such that $[[InterpR i (A {rho}) S]]$
  \end{itemize}
\end{lemma}
\begin{proof}
  The forward direction is immediate by
  Lemma~\ref{lemma:interpinv}. We now consider the backward direction
  and show that $[[G |= A : Set i]]$ given the second bullet.

  Suppose $[[rho |= G]]$, then we know that there exists some $[[S]]$
  such that $[[InterpR i (A {rho}) S]]$. By the definition of semantic
  typing, it suffices to show that there exists some $[[j]]$ and
  $[[S0]]$ such that  $[[InterpR j Set i S0]]$ and $[[A {rho} in
  S0]]$.
  Pick $[[S i]]$ for $[[j]]$ and $[[ { A | exists S , InterpR i A S }
  ]]$ for $[[S0]]$ and it's trivial to verify the conditions hold.
\end{proof}

The semantic context well-formedness judgment ($[[|= G]]$), unlike its syntactic
counterpart $[[|- G]]$, is defined through a for all quantified statement rather
than inductively over the context. It is easy to recover the same
structural rules:
\begin{lemma}[Semantic context well-formedness cons]
  \label{lemma:semwffcons}
  If $[[|= G]]$ and $[[G |= A : Set i]]$, then $[[|= G ++ A]]$.
\end{lemma}
\begin{proof}
  By the definition of semantic context well-formedness, the goal is
  to show that given $[[i < | G ++ A |]] = [[Suc |G|]]$, $[[drop Suc i
  (G ++ A) |= (G ++ A) i : Set j]]$. The statement can be easily
  proven by case analysis on whether $[[i]]$ is zero.
\end{proof}

Likewise, the semantic well-formendness judgment for substitutions
satisfies similar structural rules.
\begin{lemma}[Well-formed $[[rho]]$ cons\footnote{\dotv{soundness.v}{$\gamma$\_ok\_cons}}]
  If $[[InterpR i A S]]$, $[[a in S]]$, and $[[rho |= G]]$, then
  $[[rho .: a |= G ++ A]]$.
\end{lemma}
\begin{proof}
  Start by unfolding the definition of $[[rho .: a |= G ++ A]]$ and
  performing a case analysis similar to the proof of
  Lemma~\ref{lemma:semwffcons}.  The case where the number is $[[0]]$
  requires Lemma~\ref{lemma:logreldeterhet} to finish the proof.
\end{proof}

Next, we show some non-trivial cases of the fundamental theorem as
top-level lemmas and leave the remaining cases as exercises for the reader.
\begin{lemma}[ST-Var]
  \label{lemma:stvar}
  If $[[|= G]]$ and $[[i < |G|]]$, then $[[G |= i : up Suc i G i]]$.
\end{lemma}
\begin{proof}
  TODO: formulate renaming
\end{proof}

\begin{lemma}[ST-Set]
  \label{lemma:stset}
  If $[[i < j]]$, then $[[G |= Set i : Set j]]$.
\end{lemma}
\begin{proof}
  Immedaite by Lemma~\ref{lemma:setinv} and \rref{I-Set}.
\end{proof}

\begin{lemma}[ST-Pi]
  \label{lemma:stpi}
  If $[[G |= A : Set i]]$ and $[[G ++ A |= B : Set i]]$, then $[[G |= Pi
  A B : Set i]]$.
\end{lemma}
\begin{proof}
  Applying Lemma~\ref{lemma:setinv} to the
  conclusion, it now suffices to show that given $[[rho |= G]]$, there
  exists some $[[S]]$ such that $[[InterpR i Pi A{rho} B{up rho} S]]$.
  From Lemma~\ref{lemma:setinv} and $[[G |= A : Set i]]$, we know that
  there exists some set $[[S0]]$ such that $[[InterpR i A {rho} S0]]$.
From $[[G ++ A |= B : Set i]]$, we know that there must
exists $[[S]]$ such that $[[InterpR i B {rho .: a} S]]$ for every $[[a
in S0]]$. The conclusion immediately follows from Lemma~\ref{lemma:piintroalt}.
\end{proof}

\begin{lemma}[ST-Abs]
  \label{lemma:stabs}
  If $[[G |= Pi A B : Set i]]$ and $[[G ++ A |= b : B]]$, then $[[G |=
  \ A b : Pi A B]]$.
\end{lemma}
\begin{proof}
  By unfolding the definition of $[[G |= \ A b : Pi A B]]$, we need to
  show that given some $[[rho |= G]]$, there exists some $[[i]]$ and
  $[[S]]$ such that $[[InterpR i Pi A {rho} B {up rho} S]]$ and $[[\
  A{rho} b{up rho} in S]]$.

  By Lemma~\ref{lemma:setinv} and the premise $[[G |= Pi A B : Set
  i]]$, there exists some set $[[S]]$ such that
  $[[InterpR i Pi A {rho} B {up rho} S]]$. It now suffices to show that
  $[[\A{rho} b{up rho} in S
  ]]$. By Lemma~\ref{lemma:piinvalt}, there exists some $[[S0]]$ such
  that all following conditions hold:
  \begin{itemize}
  \item $[[InterpR i A{rho} S0]]$
  \item $[[forall a, (# a in S0 implies (# exists S1 , InterpR i B
    {rho .: a}
    S1 #) #)]]$
  \item $[[S = { b | forall a, (# a in S0 , forall
      S1, (# InterpR i B {rho .: a} S1,  b a in S1 #) #) }]]$
  \end{itemize}
  To show that $[[\A{rho} b{up rho} in S]]$, we need to prove
  that given $[[a in S0]]$,
  $[[Interp I i B {rho .: a} S1]]$, we have  $[[( \A{rho} b{up rho} )
  a in S1]]$.
  By Lemma~\ref{lemma:logrelbackclos}, the set $[[S1]]$ is closed
  under expansion. By Lemma~\ref{lemma:parabscbn}, since $[[( \A{rho} b{up rho} )
  a => b {up rho} {a}]] = [[b {rho .: a}]]$, it suffices to show that
  $[[b {rho .: a} in S1]]$, which is immediate from $[[G ++ A |= b :
  B]]$ and the fact that the logical relation is deterministic and
  cumulative (Lemma~\ref{lemma:logreldeterhet}).
\end{proof}

\begin{lemma}[ST-App]
  \label{lemma:stapp}
  If $[[G |= b : Pi A B]]$ and $[[G |= a : A]]$, then $[[G |= b a : B {a}]]$.
\end{lemma}
\begin{proof}
Suppose $[[rho |= G]]$. The goal is to show that there exists some
$[[i]]$ and $[[S1]]$
such that  $[[b {rho} a {rho} in S1 ]]$ and $[[InterpR i B {a} {rho}
S1]]$, or equivalently, $[[InterpR i B {rho .: a {rho}} S1]]$ since
$[[B {a}{rho}]] = [[B {rho .: a {rho}}]]$. By the premise $[[G |= b :
Pi A B]]$, Lemma~\ref{lemma:setinv}, and Lemma~\ref{lemma:piinvalt},
there exists some $[[i]]$ and $[[S0]]$ such that:
  \begin{itemize}
  \item $[[InterpR i A{rho} S0]]$
  \item $[[forall a0, (# a0 in S0 implies (# exists S1 , InterpR i B
    {rho .: a0}
    S1 #) #)]]$
  \item $[[forall a0, (# a0 in S0 , forall
      S1, (# InterpR i B {rho .: a0} S1,  b {rho} a0 in S1 #) #)]]$
  \end{itemize}
  Instantiating the variable $[[a0]]$ from the last two bullets with
  the term $[[a {rho}]]$, the conclusion immediately follows.
\end{proof}

\begin{theorem}[The Fundamental Theorem\footnote{\dotv{soundness.v}{soundness}}]
  \label{theorem:soundness}
  \begin{itemize}
  \item If $[[G |- a : A]]$, then $[[G |= a : A]]$.
  \item If $[[|- G]]$, then $[[|= G]]$.
  \end{itemize}
\end{theorem}
\begin{proof}
  Proof by mutual induction over the derivation of $[[G |- a :
  A]]$ and $[[|- G]]$.   The cases related to context well-formedness immediately follows
  from Lemma~\ref{lemma:semwffcons}.
Lemmas~\ref{lemma:stvar},~\ref{lemma:stset},~\ref{lemma:stpi},~\ref{lemma:stabs},~\ref{lemma:stapp}
  can be used to discharge their syntactic counterpart
  (e.g. Lemma~\ref{lemma:stabs} for case \rref{T-Abs}). The remaining
  cases not covered by the lemmas are similar to the ones already
  shown or simpler and therefore omitted from the text.

\end{proof}

\begin{corollary}[Logical Consistency]
  \label{corollary:consistency}
  The judgment $[[empty |- a : Void ]]$ is not derivable.
\end{corollary}
\begin{proof}
  Immediate from Theorem~\ref{theorem:soundness} and the $[[Void]]$ case of Lemma~\ref{lemma:interpinv}.
\end{proof}

\section{Mechanization}
\label{sec:logrelmech}
The Coq mechanization follows an almost identical structure as the
proofs presented in Section~\ref{sec:logrelproof}. The inductive
definition of the logical relation in Figure~\ref{fig:logrel} requires
the impredicativity of Coq's \texttt{Prop} sort since in \rref{I-Pi},
the function $[[F]]$ can be later instantiated into the logical
relation itself (e.g. in the proof of Lemma~\ref{lemma:piintroalt}).

\section{Related Work}
\label{sec:relatedwork}

% Type soundness can be proven through a syntactic
% approach~\citep{syntacticsoundness} as a corollary of two properties:
% progress and preservation. % The syntactic type soundness proof
% % varies in complexity depending on the underlying type
% % system. For example, a type system that tracks information flow would
% % require additional structural rules related to security levels. In
% % this paper, we focus on one specific type of complexity: the
% In Figure~\ref{fig:stlcsoundness}, we summarize the structure of the
% syntactic type soundness proof for the simply typed lambda
% calculus. Each lemma can be proven by structural induction over the
% typing derivation, while using the previous established results as
% lemmas for specific cases that do not immediately follow from the
% induction hypothesis. If we make our language more complex by adding
% full dependent type support, the overall structure remains almost
% identical.


% NbE in Coq

%%
%% The next two lines define the bibliography style to be used, and
%% the bibliography file.
\bibliographystyle{ACM-Reference-Format}
\bibliography{refs}


%%
%% If your work has an appendix, this is the place to put it.

\end{document}
\endinput
%%
%% End of file `sample-acmsmall.tex'.
